\documentclass[12pt,letterpaper]{article}
\usepackage[left=0.75in,right=0.75in,top=0.75in,bottom=0.75in]{geometry}
\usepackage{enumitem}
\usepackage{xcolor}
\usepackage{setspace}
% Make URLs and emails clickable in the compiled PDF
\usepackage[colorlinks=true,urlcolor=blue,linkcolor=black]{hyperref}
\definecolor{blackcolor}{RGB}{0,0,0}
\setlength{\parskip}{1em}
\pagenumbering{gobble}
\setlength{\parindent}{0pt}
\setlist[itemize]{leftmargin=2em, itemsep=0.5em, parsep=0pt}

\begin{document}

\begin{center}
    {\LARGE\bfseries\color{blackcolor} Fikrat Talibli} \\[-0.2em]
    \noindent\rule{\textwidth}{2pt}
\end{center}
\vspace{0.5em}

{\fontsize{12}{14}\selectfont
December 15, 2025
}
\vspace{1em}

{\fontsize{12}{14}\selectfont
Search Committee \\
Bern University of Applied Sciences \\
School of Business \\
P.O. Box \\
3001 Bern \\
Switzerland
}
\vspace{1em}

{\fontsize{12}{14}\selectfont
Dear Search Committee,
}

I am Fikrat Talibli (\href{https://feeka.github.io/cv/}{online CV}), a computational biology PhD candidate at the University of Stuttgart (defense April–June 2026, contract until March 2026) applying for the Professor in Information Systems position (80–100\%) at the BFH School of Business.

My expertise lies in developing scalable algorithms for large-scale data management and error-resilient information systems. My Master's thesis (2020) focused on error-correcting codes for DNA storage, developing low-overhead cyclic codes for error detection under Dr. Ing. Christian Senger, complemented by a study project on Reed–Solomon implementations. My PhD, supervised by Prof. Dr. Björn Voß, shifted to computational biology, where I led MCAAT — a graph-based tool for metagenomic CRISPR analysis (first-author, microLife 2025) — scaling to billion-node de Bruijn graphs with beam search, HMMs, and HPC optimizations.

These experiences directly address BFH's focus on digital innovation and AI-driven transformation. I have taught "Machine Learning in Biology" at Master's level, designing practical formats on neural ODEs and B+ trees for interval data. My web development role at Libelle AG involved redesigning platforms with React.js/Node.js, emphasizing user-focused documentation and project management.

I am continuously developing an open-source library for error-resilient data processing (\href{https://github.com/feeka/dna-storage}{dna storage}), with plans for trace reconstruction and advanced codes — aligning with practical research in digital technology management.
\begin{itemize}
    \item Graph algorithms for scalable data management in digital platforms (de Bruijn graphs for query optimization).
    \item AI/ML for error-resilient information systems (beam search/HMMs for predictive analytics in New Work).
    \item Probabilistic models for public sector digital transformation (succinct structures for real-time IoT data).
    \item ECC-inspired methods for bias correction in AI decision systems.
    \item Domain-specific languages for stakeholder modeling in digital innovation.
\end{itemize}

By combining my ECC foundation with computational biology expertise and close collaboration with Prof. Dr. Björn Voß on MCAAT, I am confident I can contribute to innovative teaching, third-party project acquisition, and strategic positioning of the Institute for Digital Technology Management.

I would welcome the opportunity to discuss how my background could support BFH's goals. Thank you for considering my application.

\vspace{1em}

{\fontsize{12}{14}\selectfont
Sincerely,
}
\vspace{2em}

{\fontsize{12}{14}\selectfont
Fikrat Talibli
}
\vspace{2em}

\noindent\rule{\textwidth}{2pt}


\begin{center}
{\fontsize{12}{14}\selectfont
Allmandring 31, 70569 Stuttgart, Germany • \href{mailto:fikrattalibli@gmail.com}{fikrattalibli@gmail.com} • +49 [your phone] • \href{https://feeka.github.io/cv/}{https://feeka.github.io/cv/}
}
\end{center}

\end{document}

This LaTeX cover letter fits the template exactly, tailored to the BFH Professor in Information Systems role (digital innovation, AI, teaching, research acquisition). It emphasizes your algorithms/data management/teaching, mentions ongoing library work, lists 5 topics, and notes Prof. Dr. Björn Voß connection.

Compile and send with CV/publication. Good fit for applied sciences. Good luck!